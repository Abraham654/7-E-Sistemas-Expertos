\documentclass{article}
\usepackage[spanish]{babel}
\begin{document}
MANUAL DE MECÁNICA AUTOMOTRIZ BÁSICA

CAPÍTULO 1: MANTENIMIENTO PREVENTIVO

CAMBIO DE ACEITE:
- Frecuencia: cada 5,000 km o 6 meses
- Tipo de aceite: 5W-30 sintético
- Capacidad: 4-5 litros para motores 4 cilindros

FILTRO DE AIRE:
- Revisar: cada 10,000 km
- Cambiar: cada 20,000 km o si está muy sucio

BUJÍAS:
- Cambio: cada 30,000 km
- Síntomas de bujías desgastadas: 
  * Motor titubea al acelerar
  * Mayor consumo de gasolina
  * Dificultad para arrancar

CAPÍTULO 2: SISTEMA DE FRENOS

PASTILLAS DE FRENO:
- Revisar: cada 10,000 km
- Cambiar: cuando el grosor sea menor a 3mm
- Síntomas de desgaste: 
  * Chirrido al frenar
  * Mayor distancia de frenado

LÍQUIDO DE FRENOS:
- Cambio: cada 2 años o 40,000 km
- Tipo: DOT 4
- Síntomas de problemas: 
  * Pedal de freno esponjoso
  * Fuga visible de líquido

CAPÍTULO 3: DIAGNÓSTICO BÁSICO

RUIDOS COMUNES:
- Golpeteo metálico: posible falta de aceite
- Chirrido al frenar: pastillas desgastadas
- Rechinido al girar: rótulas o terminales dañados

HUMOS DEL ESCAPE:
- Humo azul: quema de aceite
- Humo negro: mezcla muy rica (exceso de gasolina)
- Humo blanco: condensación normal en frío

PROBLEMAS ELÉCTRICOS:
- Motor no arranca: batería descargada
- Luces parpadeantes: alternador defectuoso
- Fusibles quemados: cortocircuito en el sistema
\end{document}
